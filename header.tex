\documentclass[11pt,twoside,openleft]{book}
\usepackage{parskip}
\usepackage{graphicx}
\usepackage{multicol}
\usepackage{geometry}
\geometry{
    letterpaper,
    left=0.5in,
    right=0.5in,
    top=0.75in,
    bottom=0.5in,
}
\setlength{\textwidth}{7.3in}

\usepackage{titletoc}
\usepackage{titlesec}
\usepackage{lscape}
\usepackage{longtable}
\usepackage{float}
\usepackage[hidelinks,bookmarks,bookmarksdepth=2]{hyperref}

% Checklist
\usepackage{enumitem,amssymb}
\newlist{fixlist}{itemize}{2}
\setlist[fixlist]{label=$\square$}
\usepackage{pifont}
\newcommand{\cmark}{\ding{51}}%
\newcommand{\xmark}{\ding{55}}%
\newcommand{\checked}{\rlap{$\square$}{\raisebox{2pt}{\large\hspace{1pt}\cmark}}%
\hspace{-2.5pt}}
\newcommand{\needfix}{\rlap{$\square$}{\large\hspace{1pt}\xmark}}

\renewcommand{\chaptermark}[1]{\markboth{\MakeUppercase{#1}}{}}

%-- Need to find a better solution than this for chapter pages
\titleformat{\chapter}[display]
{\normalfont\Large\filcenter}
{\vspace*{\fill}
 \titlerule[1pt]%
 \vspace{1pt}%
 \titlerule
 \vspace{1pc}%
 \LARGE\MakeUppercase{}~\thechapter}
{1pc}
{\titlerule\Huge}[]


%-- Deal with unicode accents in text
\include{unicode_accents}
\DeclareGraphicsExtensions{001.eps}

\setlength{\parindent}{0pt}


% -----